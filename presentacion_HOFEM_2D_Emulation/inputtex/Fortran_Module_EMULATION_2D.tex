%%%%%%%%%%%%%%%%%%%%%%%%%%%%%%%%%%%%%%%%%%%%%%%%%%%%%%%%%%%%%%%%%%%%%%%%% 
% $Id$
% %%%%%%%%%%%%%%%%%%%%%%%%%%%%%%%%%%%%%%%%%%%%%%%%%%%%%%%%%%%%%%%%%%%%%%%%%
%
% Set de slides presentando el modulo fortran y sobre todo las lineas
% relevantes del fichero de entrada que tienen que ver con
% ello incluyendo el fichero de farfield.conf.
%
% Tambien se incluyen resultados del campo lejano con solucion
% analitica
% 
% %%%%%%%%%%%%%%%%%%%%%%%%%%%%%%%%%%%%%%%%%%%%%%%%%%%%%%%%%%%%%%%%%%%%%%%%%

  \begin{frame}[fragile,allowframebreaks]{Fortran Module}{\url{EMULATION_2D_scattering_module.F90}}

    \begin{lstlisting}[style=myFORTRANcodeS]
   TYPE :: EMULATION_2D_scattering_type
      PRIVATE 
      LOGICAL :: enabled_EMULATION_2D_scattering = &
           DEFAULT_ENABLED_EMULATION_2D_SCATTERING
      CHARACTER(len=2) :: green_function_type = DEFAULT_GREEN_FUNCTION_TYPE
      LOGICAL :: enabled_normal_incidence = DEFAULT_ENABLED_NORMAL_INCIDENCE
      ! Thickness (height) of 3D slice used to emulate 2D
      REAL(KIND=DBL) :: slice_thickness = DEFAULT_SLICE_THICKNESS 
   END TYPE EMULATION_2D_scattering_type

   PUBLIC ::  &
      is_EMULATION_2D_scattering_enabled, &
      enable_EMULATION_2D_scattering, &
      disable_EMULATION_2D_scattering, &
      get_EMULATION_2D_scattering_green_function_type, &
      set_EMULATION_2D_scattering_green_function_type, &
      is_EMULATION_2D_scattering_normal_incidence, &
      enable_EMULATION_2D_scattering_normal_incidence, &
      disable_EMULATION_2D_scattering_normal_incidence, &
      get_EMULATION_2D_scattering_slice_thickness, &
      set_EMULATION_2D_scattering_slice_thickness, &
      EMULATION_2D_scattering_print, &
      EMULATION_2D_scattering_sanitycheck         
    \end{lstlisting}


   \framebreak % %%%%%%%%%%%%%%%%%%%%%%%%%%

   
   \begin{block}{Input file \url{.em}} 
     \begin{lstlisting}[basicstyle=\ttfamily\footnotesize,tabsize=3,frame=none]
--------------------------------------
--     EMULATION 2D properties      --
--------------------------------------
-- EMULATION 2D activation flag
EMULATION_2D = true
-- Set the type of green function used
EMULATION_2D_green_function_type = "2D"
     \end{lstlisting}
   \end{block}

   \begin{itemize}
   \item Once EMULATION\_2D mode is activated FE-IIEE and farfield
     calculations are specialized
   \item By default, the Green's function used is the one for 2D
     problems, i..e, $H_0^2$ (selected as above):
     \begin{center}
     \url{EMULATION_2D_green_function_type = "2D"}
   \end{center}

 \item If ``3D'' is selected, the Green's function of 3D is used and
   Ewald acceleration (``Ewald 1D'') is automatically activated.
   \end{itemize}
   
   \framebreak % %%%%%%%%%%%%%%%%%%%%%%%%%%

   
   \begin{block}{Input file \url{.em}} 
     \begin{lstlisting}[basicstyle=\ttfamily\footnotesize,tabsize=3,frame=none]
--------------------------------------
--     EMULATION 2D properties      --
--------------------------------------
-- Activation flag for normal incidence
EMULATION_2D_normal_incidence = true
-- Thickness of the slice used as pseudo2D problem
EMULATION_2D_slice_thickness = 1.0
     \end{lstlisting}
   \end{block}

   \begin{itemize}
   \item The case of ``normal incidence'' is kept as a separate case
     \begin{itemize}
     \item It does not require periodic meshes
     \end{itemize}
   \item Value of ``slice thickness'' set up manually (desirable to be
     set automatically, e.g., GUI or HOFEM mesh preprocessing)
   \end{itemize}
   
  \end{frame}
  
% %%%%%%%%%%%%%%%%%%%%%%%%%%%%%%%%%%%%%%%%%%%%%%%%%%%%%%%%%%%%%%%
  
  \begin{frame}[fragile,allowframebreaks]{Fortran Module}{\url{EMULATION_2D_scattering_module.F90} }

    \begin{block}{Input file \url{.em}} 
      \begin{lstlisting}[basicstyle=\ttfamily\footnotesize,tabsize=3,frame=none]
--------------------------------------
--     EMULATION 2D properties      --
--------------------------------------
-- Orientation of the cylinder (axis unit vector)
EMULATION_2D_cylinder_axis = {0.0,1.0,0.0}
     \end{lstlisting}
   \end{block}


   \begin{itemize}
   \item Cylinder axis in arbitrary direction
     \begin{itemize}
     \item Cylinder axis is set up manually (desirable to be set
       automatically, e.g., GUI or HOFEM mesh preprocessing)
     \end{itemize}
   \end{itemize}     

   
   \framebreak % %%%%%%%%%%%%%%%%%%%%%%%%%%%%%%%%%%%%%
    \begin{block}{Input file \url{.em}} 
      \begin{lstlisting}[basicstyle=\ttfamily\footnotesize,tabsize=3,frame=none]
--------------------------------------
--     EMULATION 2D properties      --
--------------------------------------
-- Number of incidence angles to be analized
EMULATION_2D_num_exterior_excitations = 1
-- Type of plane wave ("Single", "Multiple")
EMULATION_2D_exterior_type_1 = "Single"
-- Angle of incidence (with respect to local spherical coordinate
-- system with z axis along the cylinder axis)
--   * First number is 'theta' (angle with respect to cylinder axis;
--     90 corresponds to normal incidence)
--   * Second number is 'phi' (angle of associated local cylindrical
--     coordinate system)
EMULATION_2D_exterior_angle_1 = {90,45}
-- Polarization (TM or TE)
EMULATION_2D_polarization_type_1 = "TM"
     \end{lstlisting}
   \end{block}

   \framebreak % %%%%%%%%%%%%%%%%%%%%%%%%%%%%%%%%%%%%%

   \begin{itemize}
   \item Definition of angles for excitations with respect to local
     spherical coordinate system with z axis along the cylinder axis
     \begin{itemize}
     \item Angles are transformed back and forth between local and
       global axis. 
     \end{itemize}

   \item Definition of polarization as either ``TM'' or ``TE''
     \begin{itemize}
     \item Vector components (polarization) are transformed back and
       forth between local and global axis.
     \end{itemize}

   \end{itemize}

   \framebreak % %%%%%%%%%%%%%%%%%%%%%%%%%%%%%%%%%%%%%


   \begin{block}{Input file \url{farfield_EMULATION_2D.conf}} 
      \begin{lstlisting}[basicstyle=\ttfamily\footnotesize,tabsize=3,frame=none]
!- Farfield mode (bistatic,monostatic)
Bistatic
!- Farfield component (|rE-longitudinal|, rE-longitudinal-real, rE-longitudinal-imag, rE-longitudinal-phase, |\
rE-transverse|, rE-transverse-real, rE-transverse-imag, rE-transverse-phase, RCS, RCS-transverse, RCS-longitud\
inal, RCS-dB, RCS-longitudinal-dB, RCS-transverse-dB, Monostatic-RCS, Monostatic-RCS-dB)
RCS
!- Frequency index and number of rhs
1 1
!- Rhs index
1
!- Rhs phase
0
!- Rhs amplitude
1
     \end{lstlisting}
   \end{block}

   \framebreak % %%%%%%%%%%%%%%%%%%%%%%%%%%%%%%%%%%%%%


   \begin{block}{Input file \url{farfield_EMULATION_2D.conf}} 
      \begin{lstlisting}[basicstyle=\ttfamily\footnotesize,tabsize=3,frame=none]
!- Monitor number
1
!- Phi_cyl sampling points (num init stop)
37 0 360
!- RCS units                             
sigma-lambda
!- Array analysis (flag,num_elem_u,num_elem_v) 
0 1 1                                                                            
     \end{lstlisting}
   \end{block}

   \framebreak % %%%%%%%%%%%%%%%%%%%%%%%%%%%%%%%%%%%%%
   
   \begin{itemize}
   \item Specialization to EMULATION 2D mode
     \begin{itemize}
     \item Farfield-mode: only scattering %(no ``Antenna'' support??)
     \item Farfield-component: ``longitudinal'' and ``transverse''
       components with respect to cylinder
     \item \ldots
     \end{itemize}
     
   \item Only the angles ``phi'' need to be defined for the plot (plot
     asociated to the transversal plane to the cylinder (perpendicular
     to its axis)

   \item Array analysis: we keep it as it might be of interest to
     consider finite 1D/2D periodic structures based on pseudo2D
     cylinders

   \end{itemize}


  \end{frame}

  
% %%%%%%%%%%%%%%%%%%%%%%%%%%%%%%%%%%%%%%%%%%%%%%%%%%%%%%%%%%%%%%%


  \begin{frame}[fragile,allowframebreaks]{Fortran Module}{\url{EMULATION_2D_scattering_module.F90} }

    \begin{itemize}
    \item Near field: {GreenD} derivatives are rewritten in a compact way
      supporting computation of the scattering near field (within
      FE-IIEE loop) for arbitrarily oriented cylinders

    \item Far field: Also works with arbitrarily oriented cylinders in
      terms of local $\theta$, $\phi$ components (local in the sense
      of spherical coordinate system with axis $z$ along the cylinder
      axis)
      \begin{itemize}
      \item Note that for normal incidence
        \begin{itemize}
        \item $\theta$-component is longitudinal component
          ($-z$-component) 
        \item $\phi$-component is transverse component
        \end{itemize}
      \item Perfect agreement between numerical and analytical
        solutions for both TE/TM polarizations and normal incidence \ldots
        \begin{itemize}
        \item However, we have \alert{identified a case for oblique
            incidence which there is no agreement neither with TM or TE
            polarization} (See file \url{FarField_bug.pdf} for details)
        \end{itemize}
      \end{itemize}
    \end{itemize}

  \end{frame}


  
% %%%%%%%%%%%%%%%%%%%%%%%%%%%%%%%%%%%%%%%%%%%%%%%%%%%%%%%%%%%%%%%
