%%%%%%%%%%%%%%%%%%%%%%%%%%%%%%%%%%%%%%%%%%%%%%%%%%%%%%%%%%%%%%%%%%%%%%%%% 
% $Id$
% %%%%%%%%%%%%%%%%%%%%%%%%%%%%%%%%%%%%%%%%%%%%%%%%%%%%%%%%%%%%%%%%%%%%%%%%%
%
% Set de slides estudiando diferencias entre uso de Green3D o Green2D
% para problema cilindro infinito usando una seccion (rodaja 3D) del
% cilindro
%
% %%%%%%%%%%%%%%%%%%%%%%%%%%%%%%%%%%%%%%%%%%%%%%%%%%%%%%%%%%%%%%%%%%%%%%%%%

\begin{frame}[allowframebreaks]{HOFEM 2D Emulation}

  \begin{columns}
    \column{0.25\textwidth} \centering
    {\includegraphics[angle=0,width=\textwidth]{2D_Cylinder_sliced}}

    \column{0.7\textwidth} \centering
    \begin{block}{3D Green's Function}
      \begin{itemize}
      \item Sum of infinite contributions (from each repeated cell)
      \item Ewald acceleration needed
        \begin{itemize}
        \item It works great, though
        \end{itemize}
        \begin{equation*}
          \text{Green}(R) = \dfrac{1}{4\pi}
          \sum_{n=-\infty}^\infty \dfrac{e^{-j k R_n}}{R_n}
        \end{equation*}
        with
        \begin{equation*}
          R_n=\abs{\vr-\vrp_n}, \vrp_n=\vrp_0 + n d
        \end{equation*}
      \end{itemize}
    \end{block}
  \end{columns}

  \framebreak % %%%%%%%%%%%%%%%%%%%%%%%%%%%%%%%%%%%%%%

  \begin{columns}
    \column{0.25\textwidth} \centering
    {\includegraphics[angle=0,width=\textwidth]{2D_Cylinder_unitslice}}
    
    \column{0.7\textwidth} \centering
    \begin{block}{2D Green's Function}
      \begin{itemize}
      \item No summation needed
        \begin{equation*}
          \text{Green}(\rho) = \dfrac{1}{4\im}  H_0^{(2)}(k|\rho|)
        \end{equation*}
      \end{itemize}
    \end{block}
  \end{columns}

  
 \end{frame}
  
% %%%%%%%%%%%%%%%%%%%%%%%%%%%%%%%%%%%%%%%%%%%%%%%%%%%%%%%%%%%%%%%

\begin{frame}[allowframebreaks]{Near Field (FE-IIEE loop)}
  
  \begin{block}{Sanity Checks (MATLAB)}
    \begin{itemize}
    \item Equivalence of infinite sum of 3D Green's function with 2D
      Green's function on the unit cell 
  \end{itemize}
    \begin{columns}%[T]
      \column{0.48\textwidth}
      \includegraphics[width=\textwidth]{GreenFunctions_R.pdf} \\
     % \footnotesize{Green's function values as a function of $\rho$}
      \column{0.48\textwidth}
      \includegraphics[width=\textwidth]{GreenFunctions_convN_reim.pdf} \\
      %\footnotesize{3D Green's function values as the number of unit
      %  cells considered (no Ewald acceleration)}
    \end{columns}
  \end{block}

  \framebreak % %%%%%%%%%%%%%%%%%%%%%%%%%%%%%%%%%%%%%%

  \begin{block}{Sanity Checks (MATLAB)}
    Equivalence of infinite sum of 3D Green's function with 2D Green's
    function on the unit cell
    \begin{columns}%[T]
      \column{0.48\textwidth}
       \includegraphics[width=\textwidth]{GreenFunctions_Comp.pdf} \\
%      \footnotesize{Difference Green's function values (3D summation and 2D)
%        as a function of $\rho$}
      \column{0.48\textwidth}
      \includegraphics[width=\textwidth]{GreenFunctions_convN_abs.pdf} \\
%      \footnotesize{Difference Green's function values (3D summation
%        and 2D) with the number of unit cells (slices) considered (no
%        Ewald acceleration)}
    \end{columns}
  \end{block}

  \framebreak % %%%%%%%%%%%%%%%%%%%%%%%%%%%%%%%%%%%%%%

  \begin{block}{HOFEM Implementation}

    \begin{itemize}
    \item Coded
    \item Tested
    \end{itemize}
    
    \begin{columns}%[T]
      \column{0.45\textwidth}
      \lstinputlisting[basicstyle=\tiny,tabsize=3,frame=none,
      % caption={3D Green's Function},
      firstline=125,lastline=145]{HOFEM_output_3D_greens_function.txt}
      
      % \column{0.45\textwidth}
      % \lstinputlisting[basicstyle=\tiny,tabsize=3,frame=none,caption={2D
      %   Green's Function},firstline=125,lastline=130]{HOFEM_output_2D_greens_function.txt}
      
    \end{columns}
  \end{block}

  % \framebreak % %%%%%%%%%%%%%%%%%%%%%%%%%%%%%%%%%%%%%%

  % \lstinputlisting[basicstyle=\tiny,tabsize=3,frame=none,caption={2D
  %   Green's Function},firstline=20,lastline=30]{HOFEM_output_diff_3D_2D_greens_function_sidebyside.txt}

  \framebreak % %%%%%%%%%%%%%%%%%%%%%%%%%%%%%%%%%%%%%%

 Code output (terminal):
  
 \lstinputlisting[basicstyle=\scriptsize,tabsize=3,frame=none,
 % caption={2D Green's Function}
 ]{HOFEM_output_diff_3D_2D_greens_function_iterations.txt}


  \framebreak % %%%%%%%%%%%%%%%%%%%%%%%%%%%%%%%%%%%%%%


  \begin{columns}
    \column{0.48\textwidth} \centering
    {\includegraphics[angle=0,width=\textwidth]{Cylinder_0_5b_nearfield_3D_greens}}
    \column{0.48\textwidth} \centering
    {\includegraphics[angle=0,width=\textwidth]{Cylinder_0_5b_nearfield_3D_greens}}
  \end{columns}    

  
\end{frame}


% %%%%%%%%%%%%%%%%%%%%%%%%%%%%%%%%%%%%%%%%%%%%%%%%%%%%%%%%%%%%%%%

\begin{frame}[allowframebreaks]{Far Field (postprocess)}

  \begin{block}{Equivalence of Green's Functions for Far Field}
    \begin{itemize}
    \item Can we use an infinite sum of 3D Green's far-field functions
      to model the 2D Green's far-field?
      \begin{itemize}
      \item \alert {NO}
      \item and it makes sense!
      \end{itemize}
    \end{itemize}
  \end{block}
    

  \begin{block}{Far Field as Fourier Transform} 
      The Fourier Transform of a constant current contribution along
      $z$ is a delta function in the spatial frequency domain, i.e.,
      variable $\theta$:
      % 
      \begin{equation}
        \text{Far Field}(\theta,\phi) =  \alert{\delta(\theta-\pi/2)}  F(\phi)
      \end{equation}
      
    \end{block}
  
    \framebreak % %%%%%%%%%%%%%%%%%%%%%%%%%%%%%%%%%%%%%%

    \begin{columns}
      \column{0.46\textwidth} \centering
      \begin{block}{$\theta\ne 90^\circ$: \alert{Oscillations}}
        \includegraphics[clip=true,trim=100 200 100 200,width=\textwidth]{GreenFar_abs_theta45} \\
      \end{block}
      
      \column{0.46\textwidth} \centering
      \begin{block}{$\theta=90^\circ$: \alert{Divergence}}
        \includegraphics[clip=true,trim=100 200 100 200,width=\textwidth]{GreenFar_abs_theta90} \\
      \end{block}
            
    \end{columns}

    \framebreak % %%%%%%%%%%%%%%%%%%%%%%%%%%%%%%%%%%%%%%

  \begin{block}{HOFEM Implementation}

    \begin{itemize}
    \item  Coded (in progress)
    \item Tested (in progress)
    \end{itemize}
  \end{block}  
    
  \end{frame}

  % %%%%%%%%%%%%%%%%%%%%%%%%%%%%%%%%%%%%%%%%%%%%%%%%%%%%%%%%%%%%%%%

  \begin{frame}[plain]
    \centering    \Large{Rethinking\ldots}

    \vspace*{0.5\baselineskip}

    \centering\parbox{\textwidth}{%
      AIRBUS: \emph{\foreignlanguage{spanish}{hay un chico con nosotros que
          resuelve problemas de scattering de cilindros y no usa
          función de Hankel. Usa Green3D para calcular el campo
          (lejano) de scattering de una rodaja y \alert{le funciona.}}
        }}

      \vspace*{\baselineskip}
      
      \begin{itemize}
      \item Further study of differences between  {\GreenD} and {\GreenT}
        \begin{itemize}
        \item Obvious (after a maturation process) for far field
        \item Not so obvious for near field (i.e., in its impact in
          FE-IIEE loop)
        \end{itemize}
      \end{itemize}
      
  \end{frame}
  

% %%%%%%%%%%%%%%%%%%%%%%%%%%%%%%%%%%%%%%%%%%%%%%%%%%%%%%%%%%%%%%%

\begin{frame}[allowframebreaks]{Far Field (postprocess)}

  \begin{block}{Equivalence of Green's Functions for Far Field}
    \begin{itemize}
    \item Can we use an infinite sum of 3D Green's far-field functions
      to model the 2D Green's far-field?
      \begin{itemize}
      \item NO
      \item and it makes sense!
      \item No need of infinite sum (\alert{far field {\GreenT}/{\GreenD} is enough})
        \begin{itemize}
        \item Note that far field versions of {\GreenT} (with
          $r=\rho$) and {\GreenD} are equal up to a constant
        \end{itemize}

      \end{itemize}
    \end{itemize}
  \end{block}
    

  \begin{block}{Far Field as Fourier Transform} 
      The Fourier Transform of a constant current contribution along
      $z$ is a delta function in the spatial frequency domain, i.e.,
      variable $\theta$:
      % 
      \begin{equation}
        \text{Far Field}(\theta,\phi) =  \alert{\delta(\theta-\pi/2)}  F(\phi)
      \end{equation}
      
    \end{block}
  
    \framebreak % %%%%%%%%%%%%%%%%%%%%%%%%%%%%%%%%%%%%%%
    
    \begin{block}{{\GreenD} vs {\GreenT}}

      
      $F(\phi)$ is the same (after normalization) considering asymptotic
      expressions of {\GreenT} and {\GreenT}
      
    \begin{columns}\centering
      \column{0.4\textwidth}% \centering
      %
      \begin{equation*}
        H_0^{(2)}(k\rho)
        \enskip \overset{\rho\rightarrow\infty}{\approx} \enskip
        \sqrt{\dfrac{2 j }{\pi k \rho}} \, e^{-jk\rho}
      \end{equation*}

      \column{0.4\textwidth}% \centering
      % 
      % 
      \begin{equation*}
        \dfrac{e^{-jkR}}{4\pi R}
         \enskip \overset{r\rightarrow\infty}{\approx}  \enskip
        \dfrac{e^{-jkr}}{4\pi r} e^{j k r^\prime\! \cos\psi}
      \end{equation*}

    \end{columns}
    
    \end{block}


  \end{frame}

  
% %%%%%%%%%%%%%%%%%%%%%%%%%%%%%%%%%%%%%%%%%%%%%%%%%%%%%%%%%%%%%%%

\begin{frame}[allowframebreaks]{Near Field (FE-IIEE loop)}
  
% {\Gree3D vs {\GreenD} }


 Code output (terminal):

 \begin{itemize}
 \item From previous meeting % Large $S'-S$
   \begin{columns}%[T]
   \column{0.45\textwidth} {\GreenD} 
   \column{0.45\textwidth} {\GreenT} 
 \end{columns}   

 \lstinputlisting[basicstyle=\scriptsize,tabsize=3,frame=none,
 % caption={2D Green's Function}
 ]{HOFEM_output_diff_3D_2D_greens_function_iterations_reversed.txt}

 \vspace{\baselineskip}

 \alert{How can it work?} Note we used {\GreenT} itself (only one
 slide was considered)
 
 
 \framebreak % %%%%%%%%%%%%%%%%%%%%%%%%%%%%%%%%%%%%%%

 Code output (terminal):

 \item Large $S'-S$
   \begin{columns}[T]
   \column{0.45\textwidth} {\GreenD} 
   \column{0.45\textwidth} {\GreenT} 
 \end{columns}   

 \lstinputlisting[basicstyle=\scriptsize,tabsize=3,frame=none,
 ]{HOFEM_output_diff_3D_2D_greens_function_iterations_reversed.txt}
 
\item Small $S'-S$
   \begin{columns}%[T]
   \column{0.45\textwidth} {\GreenD} 
   \column{0.45\textwidth} {\GreenT} 
 \end{columns}   

 \lstinputlisting[basicstyle=\scriptsize,tabsize=3,frame=none,
 ]{HOFEM_output_diff_3D_2D_greens_function__small_S-Sp_iterations_reversed.txt}

 
\end{itemize}

\end{frame}

  % %%%%%%%%%%%%%%%%%%%%%%%%%%%%%%%%%%%%%%%%%%%%%%%%%%%%%%%%%%%%%%% 

% \begin{frame}[allowframebreaks]{Study of {\GreenT} vs {\GreenD}}
 
 
%   \textbf{GRAFICAS DE LAS GEOMETRIAS USADAS EN LAS PRUEBAS}


%   \begin{itemize}
%   \item Lo que le enseñamos meeting pasado fue sobre (S'-S grande,
%     espesor pequeño y S,S' rectangulares):
    
% \begin{verbatim}
% |`cylinder_05b_1`               |1      |1      |0.05     |Rectangular    |Rectangular      |1
% \end{verbatim}
    
%     \item Durante el finde (S'-S pequeño y espesor grande, con S', S circulares)
  
% \begin{verbatim}
%     |`cylinder_05e_01_01_075`       |0.1    |0.1    |1        |Circular       |Circular         |0.75
% \end{verbatim}

%     \item Para ver efecto del espesor se puede comparar 
      
% \begin{verbatim}
%     |`cylinder_05e_01_01_075`       |0.1    |0.1    |1        |Circular       |Circular         |0.75
%     |`cylinder_sp_circled_s_circled_d_1_h_005_m_075`  |0.1|0.1|0.05   |Circular    |Circular     |0.75
% \end{verbatim}

%     \item Mallado con S' sobre PEC para paranoias de J,M equivalentes.

%       Uno de los:
% \begin{verbatim}
% Espesor bajo, distancia S-S' baja:
% |`cylinder_sp_pec_s_circled_d_01_h_005_m_07`  |0  |0.1|0.05   |Circular |Circular     |0.7


% Espesor alto, distancia S-S' baja:
% |`cylinder_sp_pec_s_circled_d_01_h_1_m_07`    |0  |0.1|1      |Circular |Circular     |0.7
% \end{verbatim}

      
      
%   \end{itemize}
% \end{frame}

  % %%%%%%%%%%%%%%%%%%%%%%%%%%%%%%%%%%%%%%%%%%%%%%%%%%%%%%%%%%%%%%% 

\begin{frame}[allowframebreaks]{{\GreenD} vs {\GreenT}}{Comparison
    with analytical solution}

  \begin{itemize}
  \item Most results displayed correspond to TM and E field
  \item H field (i.e., RotE) identical conclusions
  \item By duality (satisfied by HOFEM) TE results ``should'' be
    identical (tested).
  \item Most results shown here correspond to both $S'$ and $S$
    conformal to the cylinder (circular boundary). Analogous
    results/conclusions are obtained with rectangular boundaries for
    $S'$ and $S$ (and combinations of them).
  \end{itemize}
  % \textbf{\url{cylinder_05e_01_01_075} ¿Graficas del estilo de??}

  \framebreak % %%%%%%%%%%%%%%%%%%%%%%%%%%%%

  
    \begin{columns}
        \column{0.55\textwidth}
    \includegraphics[width=\linewidth]
    {results/2D/20/300/\meshCCC{01}{1}{075}/geometry.pdf}
        \column{0.4\textwidth}
        \begin{itemize}
          \item 300\,MHz
          \item Small $S-S'$
          \item Thick slice
        \end{itemize}
    \end{columns}

    \framebreak

    Evolution scattered electric field ($S'$ over $S$)

    \vbs

    Iteration \makebox[0pt][l]{1:}\hspace{2em}
    \raisebox{\baselineskip-\height}{
      \includegraphics[width=0.3\textwidth]
      {results/2D/1/300/\meshCCC{01}{1}{075}/E_S.pdf}
      \hspace{1cm}
      \includegraphics[width=0.3\textwidth]
      {results/3D/1/300/\meshCCC{01}{1}{075}/E_S.pdf}
    }

    \vbs

    Iteration \makebox[0pt][l]{20:}\hspace{2em}
    \raisebox{\baselineskip-\height}{
      \includegraphics[width=0.3\textwidth]
      {results/2D/20/300/\meshCCC{01}{1}{075}/E_S.pdf}
      \hspace{1cm}
      \includegraphics[width=0.3\textwidth]
      {results/3D/20/300/\meshCCC{01}{1}{075}/E_S.pdf}
    }

    
    \framebreak
    Evolution scattered magnetic field ($S'$ over $S$)

    \vbs

    Iteration \makebox[0pt][l]{1:}\hspace{2em}
    \raisebox{\baselineskip-\height}{
      \includegraphics[width=0.3\textwidth]
      {results/2D/1/300/\meshCCC{01}{1}{075}/H_S.pdf}
      \hspace{1cm}
      \includegraphics[width=0.3\textwidth]
      {results/3D/1/300/\meshCCC{01}{1}{075}/H_S.pdf}
    }

    \vbs

    Iteration \makebox[0pt][l]{20:}\hspace{2em}
    \raisebox{\baselineskip-\height}{
      \includegraphics[width=0.3\textwidth]
      {results/2D/20/300/\meshCCC{01}{1}{075}/H_S.pdf}
      \hspace{1cm}
      \includegraphics[width=0.3\textwidth]
      {results/3D/20/300/\meshCCC{01}{1}{075}/H_S.pdf}
    }


    \framebreak

    Evolution electric field on $S'$

    \vbs

    Iteration \makebox[0pt][l]{1:}\hspace{2em}
    \raisebox{\baselineskip-\height}{
      \includegraphics[width=0.3\textwidth]
      {results/2D/1/300/\meshCCC{01}{1}{075}/E_Sp.pdf}
      \hspace{1cm}
      \includegraphics[width=0.3\textwidth]
      {results/3D/1/300/\meshCCC{01}{1}{075}/E_Sp.pdf}
    }

    \vbs

    Iteration \makebox[0pt][l]{20:}\hspace{2em}
    \raisebox{\baselineskip-\height}{
      \includegraphics[width=0.3\textwidth]
      {results/2D/20/300/\meshCCC{01}{1}{075}/E_Sp.pdf}
      \hspace{1cm}
      \includegraphics[width=0.3\textwidth]
      {results/3D/20/300/\meshCCC{01}{1}{075}/E_Sp.pdf}
    }

    \framebreak

    \begin{itemize}
    \item Higher error with {\GreenT}
      \begin{itemize}
      \item This is a case with small distance $S-S'$ and large thickness
      \item Nevertheless, the error is always significantly higher
        than {\GreenD}
      \end{itemize}
    \item {\GreenT} does not converge to right solution
    \item Components $E_x$, $E_z$, $H_y$ are not null \\
      \footnotesize{(issue of \alert{($\ast$)} certainly added confusion on the
      debugging/verification of the code)}
      \begin{itemize}
      \item Much higher levels with {\GreenT}
      \item {\GreenT} (the Green's function itself) generates non null
        $E_x$, $E_z$, $H_y$ from $E_z$ on $S'$
      \item {\GreenD} (the Green's function itself) does not generate
        any $E_x$, $E_z$, $H_y$
        \begin{itemize}
        \item Numerically, non-zero levels of $E_x$, $E_z$, $H_y$ are
          generated because numerical FEM solution has non zero levels of
          $E_x$, $E_z$, $H_y$ (tested!)
        \end{itemize}
      \end{itemize}
    \item Numerical noise always present  due to discretization 
      \begin{itemize}
      \item Clearly visible in this case
      \item Decreases with finer discretization ---FEM mesh--- (tested!)
      \end{itemize}
    \end{itemize}
    
    {\footnotesize \alert{($\ast$)} Issue of GiD/HOFEM-GUI representation of fields on surface}
  
\end{frame}

% %%%%%%%%%%%%%%%%%%%%%%%%%%%%%%%%%%%%%%%%%%%%%%%%%%%%%%%%%%%%%%%

% Esto es un tema que no esta relacionado directamente con 2D
% Emulation sino genral del código pero se lo cuento aqui a Airbus
% porque viene a cuento.

\begin{frame}[plain]
  \centering    \Large{Rethinking\ldots}
  
  {\footnotesize \alert{($\ast$)} Issue of GiD/HOFEM-GUI representation of fields on surface}
  
\end{frame}

% %%%%%
  
\begin{frame}[allowframebreaks]{GiD/HOFEM-GUI Field Representation 
    on Surfaces}{Examples with PEC surfaces}

  \begin{itemize}
  \item $E_z$ should be null on the  caps (top and
    bottom $y=\text{cte}$)

    \vbs
   
    \begin{columns}%[T]
      \centering
      \column{0.48\textwidth}\centering
      \includegraphics[width=\textwidth,clip=true,trim=70 0 350 140]
      {Ez_original} \\
      \footnotesize{$\abs{E_z}$ (default color scale)}
      \column{0.48\textwidth}\centering
      \includegraphics[width=\textwidth,clip=true,trim=70 0 350 140]
      {Ez_original_saturado_01} \\
      \footnotesize{$\abs{E_z}$ (saturated color scale)}
    \end{columns}
    

    \framebreak % %%%%%%%%%%%%%

  \item $E_y$ should be null on the cylinder ``sides'' (i.e., on the
    PEC cylinder)

    \vbss
   
    \begin{center}
     \begin{columns}%[T]
%      \column{0.48\textwidth}\centering
      % \includegraphics[width=\textwidth,clip=true,trim=70 0 350 140]
      % {Ey_original} \\
      % \footnotesize{$\abs{E_y}$ }
      \column{0.70\textwidth}\centering
      \includegraphics[width=\textwidth,clip=true,trim=70 0 350 140]
      {Ey_internal_cylinder} \\
      \footnotesize{$\abs{E_y}$ (only PEC cylinder surface shown)}
    \end{columns}
  \end{center}
  \end{itemize}
\end{frame}

% %%%%%%%%%%%%%%%%%%%%%%%%%%%%%%%%%%%%%%%%%%%%%%%%%%%%%%%%%%%%%%%

\begin{comment} % NO ES MUY CLARA LA CONCLUSION (por tema de como sale la Green3D o el campo  H) ASI QUE SE OMITEN

\begin{frame}[allowframebreaks]{Effect of Discretization}{Simple test
    changing frequency \alert{(note other electrical distances also
      change)}}


%    $300\,\text{MHz}$ \makebox[0pt][l]{20:}\hspace{2em}
    \raisebox{\baselineskip-\height}{
      \includegraphics[width=0.3\textwidth]
      {results/2D/20/300/\meshCCC{01}{1}{075}/E_S.pdf}
      \hspace{1cm}
      \includegraphics[width=0.3\textwidth]
      {results/2D/20/300/\meshCCC{01}{1}{075}/H_S.pdf}
    }

    
    \vbs

%    Lower frequency \makebox[0pt][l]{1:}\hspace{2em}
    \raisebox{\baselineskip-\height}{
      \includegraphics[width=0.3\textwidth]
      {results/2D/20/30/\meshCCC{01}{1}{075}/E_S.pdf}
      \hspace{1cm}
      \includegraphics[width=0.3\textwidth]
      {results/2D/20/30/\meshCCC{01}{1}{075}/H_S.pdf}
    }

  \end{frame}
  
\end{comment} % %%%%%%%%%%%
  
  % %%%%%%%%%%%%%%%%%%%%%%%%%%%%%%%%%%%%%%%%%%%%%%%%%%%%%%%%%%%%%%% 

\begin{comment} % NO ES MUY CLARA LA CONCLUSION (la quitamos)

\begin{frame}[allowframebreaks]{Effect of Distance $S'-S$}{Scattered near field}

    \begin{columns}
      \column{0.25\textwidth}
        \includegraphics[width=\linewidth]
        {results/2D/20/300/\meshCCC{01}{1}{075}/geometry.pdf}

        \includegraphics[width=\linewidth]
        {results/2D/20/300/\meshCCC{1}{1}{075}/geometry.pdf}
      \column{0.3\textwidth}
        
        \includegraphics[width=\linewidth]
        {results/2D/20/300/\meshCCC{01}{1}{075}/E_S.pdf}
        
        \includegraphics[width=\linewidth]
        {results/2D/20/300/\meshCCC{1}{1}{075}/E_S.pdf}

      
    \column{0.3\textwidth}
      \hfill\GreenT\hfill\mbox{}

        \includegraphics[width=\linewidth]
        {results/3D/20/300/\meshCCC{01}{1}{075}/E_S.pdf}
        
        \includegraphics[width=\linewidth]
        {results/3D/20/300/\meshCCC{1}{1}{075}/E_S.pdf}
        

    \end{columns}

    \begin{itemize}
    \item  {\GreenT} results always worse than {\GreenD}
    \item Difficult comparison ($S$ is not at the same electrical distance) 
    \end{itemize}
    
  \end{frame}

\end{comment}  % %%%%%

  % %%%%%%%%%%%%%%%%%%%%%%%%%%%%%%%%%%%%%%%%%%%%%%%%%%%%%%%%%%%%%%% 

\begin{comment} % Idem que anterior

  \begin{frame}[allowframebreaks]{Effect of Distance $S'-S$}{FEM solution}

    \begin{columns}
      \column{0.25\textwidth}

        \includegraphics[width=\linewidth]
        {results/2D/20/300/\meshCCC{01}{1}{075}/geometry.pdf}

        \includegraphics[width=\linewidth]
        {results/2D/20/300/\meshCCC{1}{1}{075}/geometry.pdf}
      \column{0.3\textwidth}
      \hfill\GreenD\hfill\mbox{}
        
        \includegraphics[width=\linewidth]
        {results/2D/20/300/\meshCCC{01}{1}{075}/E_Sp.pdf}
        
        \includegraphics[width=\linewidth]
        {results/2D/20/300/\meshCCC{1}{1}{075}/E_Sp.pdf}

      
    \column{0.3\textwidth}
      \hfill\GreenT\hfill\mbox{}

        \includegraphics[width=\linewidth]
        {results/3D/20/300/\meshCCC{01}{1}{075}/E_Sp.pdf}
        
        \includegraphics[width=\linewidth]
        {results/3D/20/300/\meshCCC{1}{1}{075}/E_Sp.pdf}
        

    \end{columns}

    \begin{itemize}
      \item The IIEE iterative method has low effect when $S-S'$ is larger.
    \end{itemize}
    
  \end{frame}
  
\end{comment}  % %%%%%
  
  % %%%%%%%%%%%%%%%%%%%%%%%%%%%%%%%%%%%%%%%%%%%%%%%%%%%%%%%%%%%%%%% 
  
\begin{frame}[allowframebreaks]{Effect of ``Thickness''}{Scattered
    near field ---\alert{Note error in figure captions: thickness is
      $0.05\,\lambda$ for the thin slice---}}
    
    \begin{columns}
      \column{0.25\textwidth}
        \includegraphics[width=\linewidth]
        {results/2D/20/300/\meshCCC{01}{1}{075}/geometry.pdf}

        \includegraphics[width=\linewidth]
        {results/2D/20/300/\meshCCC{01}{005}{075}/geometry.pdf}
      \column{0.3\textwidth}
      \hfill\GreenD\hfill\mbox{}
        
        \includegraphics[width=\linewidth]
        {results/2D/20/300/\meshCCC{01}{1}{075}/E_S.pdf}
        
        \includegraphics[width=\linewidth]
        {results/2D/20/300/\meshCCC{01}{005}{075}/E_S.pdf}

      
    \column{0.3\textwidth}
      \hfill\GreenT\hfill\mbox{}

        \includegraphics[width=\linewidth]
        {results/3D/20/300/\meshCCC{01}{1}{075}/E_S.pdf}
        
        \includegraphics[width=\linewidth]
        {results/3D/20/300/\meshCCC{01}{005}{075}/E_S.pdf}
        

    \end{columns}

  \end{frame}

  \begin{frame}[allowframebreaks]{Effect of ``Thickness''}{FEM solution
    ---\alert{Note error in figure captions: thickness is
      $0.05\,\lambda$ for the thin slice---}}
    
    \begin{columns}
      \column{0.25\textwidth}
        \includegraphics[width=\linewidth]
        {results/2D/20/300/\meshCCC{01}{1}{075}/geometry.pdf}

        \includegraphics[width=\linewidth]
        {results/2D/20/300/\meshCCC{01}{005}{075}/geometry.pdf}
      \column{0.3\textwidth}
      \hfill\GreenD\hfill\mbox{}
        
        \includegraphics[width=\linewidth]
        {results/2D/20/300/\meshCCC{01}{1}{075}/E_Sp.pdf}
        
        \includegraphics[width=\linewidth]
        {results/2D/20/300/\meshCCC{01}{005}{075}/E_Sp.pdf}

      
    \column{0.3\textwidth}
      \hfill\GreenT\hfill\mbox{}

        \includegraphics[width=\linewidth]
        {results/3D/20/300/\meshCCC{01}{1}{075}/E_Sp.pdf}
        
        \includegraphics[width=\linewidth]
        {results/3D/20/300/\meshCCC{01}{005}{075}/E_Sp.pdf}
        

    \end{columns}

    \framebreak 
    
    \begin{itemize}
    \item The thickness is the main factor for error with {\GreenT}
    \item {\GreenD} behaves great!
    \item Always lower error  with thin slices% (although convergence is slow)
    \end{itemize}
    
    
  \end{frame}

    

  % %%%%%%%%%%%%%%%%%%%%%%%%%%%%%%%%%%%%%%%%%%%%%%%%%%%%%%%%%%%%%%% 
  

  \begin{comment} %%%%%%%  POR HACER
    
  \begin{frame}[allowframebreaks]{Effect on Polarization Purity}

    \begin{itemize}
    \item With ``thickness'' \textbf{Usamos espesor gordo para que se
        note:} Por ejemplo, O bien
      \url{cylinder_sp_circled_s_circled_d_1_h_1_m_075} o bien
      \url{cylinder_sp_circled_s_circled_d_01_h_1_m_075} (no se en
      cual se notará mas según distancia S'-S, uhm.....)

      \vskip 1em
      
      \begin{columns}%[T]
      \column{0.48\textwidth}
      \FIGURA{\GreenT}
%      \includegraphics[width=\textwidth]{kk_iter1} \\
%      \footnotesize{kkkkk}
      \column{0.48\textwidth}
      \FIGURA{\GreenD}
%      \includegraphics[width=\textwidth]{kk_iter82} \\
      %\footnotesize{kkkk}
    \end{columns}
  \end{itemize}
    \begin{itemize}
    \item Conclusion 1
    \item \ldots
    \end{itemize}
    
  \end{frame}

  \end{comment} %%%%%%%%
  
  % %%%%%%%%%%%%%%%%%%%%%%%%%%%%%%%%%%%%%%%%%%%%%%%%%%%%%%%%%%%%%%% 
  
  \begin{frame}[allowframebreaks]{Other Effects (paranoic
      mode)}{Effect of $S'$ on PEC cylinder}

    \begin{columns}
    \column{0.25\textwidth}

        \includegraphics[width=\linewidth]
        {results/3D/20/300/\meshCCC{1}{1}{075}/geometry.pdf}
        
        \includegraphics[width=\linewidth]
        {results/3D/20/300/\meshPC{1}{1}{085}/geometry.pdf}
      \column{0.3\textwidth}

        \includegraphics[width=\linewidth]
        {results/2D/20/300/\meshCCC{1}{1}{075}/E_S.pdf}

        \includegraphics[width=\linewidth]
        {results/2D/20/300/\meshPC{1}{1}{085}/E_S.pdf}
      \column{0.3\textwidth}
        
        \includegraphics[width=\linewidth]
        {results/2D/20/300/\meshCCC{1}{1}{075}/H_S.pdf}
        
        \includegraphics[width=\linewidth]
        {results/2D/20/300/\meshPC{1}{1}{085}/H_S.pdf}

    \end{columns}
    
  \end{frame}


  % %%%%%%%%%%%%%%%%%%%%%%%%%%%%%%%%%%%%%%%%%%%%%%%%%%%%%%%%%%%%%%% 
  
  \begin{frame}[allowframebreaks]{Other Effects (paranoic
      mode)}{Effect of $S'$, $S$ Curved}
    
      \begin{columns}
    \column{0.25\textwidth}

        \includegraphics[width=\linewidth]
        {results/3D/20/300/\meshCCC{01}{1}{075}/geometry.pdf}
        
        \includegraphics[width=\linewidth]
        {results/3D/20/300/\meshCSS{01}{1}{075}/geometry.pdf}
      \column{0.3\textwidth}

        \includegraphics[width=\linewidth]
        {results/2D/20/300/\meshCCC{01}{1}{075}/E_S.pdf}

        \includegraphics[width=\linewidth]
        {results/2D/20/300/\meshCSS{01}{1}{075}/E_S.pdf}
      \column{0.3\textwidth}
        
        \includegraphics[width=\linewidth]
        {results/2D/20/300/\meshCCC{01}{1}{075}/H_S.pdf}
        
        \includegraphics[width=\linewidth]
        {results/2D/20/300/\meshCSS{01}{1}{075}/H_S.pdf}

      
        

    \end{columns}
    

    
  \end{frame}

% %%%%%%%%%%%%%%%%%%%%%%%%%%%%%%%%%%%%%%%%%%%%%%%%%%%%%%%%%%%%%%%

  \begin{frame}[allowframebreaks]{{\GreenT} vs {\GreenD}}{Conclusions}

    \begin{block}{Conclusions}
      \begin{itemize}
      \item {\GreenT} itself not valid for 2D Emulation
        \begin{itemize}
        \item FE-IIEE converges but
          \begin{itemize}
          \item Quality of solution far from great
          \item It may guide you to wrong solution
          \end{itemize}
        \end{itemize}
      \item Need to activate contributions of slice replicas along
        direction of cylinder (translation symmetry) \alert{\GreenTEw}
      \end{itemize}
    \end{block}
  
  \end{frame}

% %%%%%%%%%%%%%%%%%%%%%%%%%%%%%%%%%%%%%%%%%%%%%%%%%%%%%%%%%%%%%%%%%%%  

  \begin{frame}[plain]
    From now on we only use {\GreenD}
    \begin{itemize}
    \item We check TE polarization for PEC cylinder
    \item We check for the dielectric coated  cylinder 
    \end{itemize}
  \end{frame}
  
% %%%%%%%%%%%%%%%%%%%%%%%%%%%%%%%%%%%%%%%%%%%%%%%%%%%%%%%%%%%%%%%
  
  \begin{frame}[allowframebreaks]{Check TE polarization}{Comparison
      with TM results}
    
    \begin{columns}
      \column{0.5\linewidth}
      \includegraphics[width=\linewidth]{results/TE/geometry.pdf}
      
      \column{0.45\linewidth}
      \begin{itemize}
      \item
        Freq: 300\,MHz. ($\lambda=1\,$m)
      \item
        Thickness: 0.05\,m
      \end{itemize}
      
    \end{columns}
    
    \framebreak
    
    \begin{columns}
      \column{0.35\linewidth}
      \hfill TM incident field \hfill\mbox{}
      \vspace{1ex}
      \includegraphics[width=\linewidth]{results/TM/E_Sp.pdf}
      \includegraphics[width=\linewidth]{results/TM/E_S.pdf}
      
      \column{0.35\linewidth}
      \hfill TE incident field \hfill\mbox{}
      \vspace{1ex}
      \includegraphics[width=\linewidth]{results/TE/E_Sp.pdf}
      \includegraphics[width=\linewidth]{results/TE/E_S.pdf}
      
    \end{columns}
    
    \framebreak
    
    \begin{columns}
      \column{0.35\linewidth}
      \hfill TM incident field \hfill\mbox{}
      \vspace{1ex}
      \includegraphics[width=\linewidth]{results/TM/H_Sp.pdf}
      \includegraphics[width=\linewidth]{results/TM/H_S.pdf}
      
      \column{0.35\linewidth}
      \hfill TE incident field \hfill\mbox{}
      \vspace{1ex}
      \includegraphics[width=\linewidth]{results/TE/H_Sp.pdf}
      \includegraphics[width=\linewidth]{results/TE/H_S.pdf}
      
    \end{columns}
    
  \end{frame}

  % %%%%%%%%%%%%%%%%%%%%%%%%%%%%%%%%%%%%%%%%%%%%%%%%%%%%%%%%%%%%%%%%%%%%
  
  \begin{frame}[allowframebreaks]{Dielectric Coated Cylinders}{Comparison with the analytical solution }

    
    \begin{columns}
      \column{0.5\linewidth}
      \includegraphics[width=\linewidth]{results/TMc4/geometry.pdf}
      
      \column{0.45\linewidth}
      \begin{itemize}
      \item
        Freq: 300\,MHz. ($\lambda=1\,$m)
      \item
        Thickness: 0.05\,m
      \item
        Dielectric substrate between $S$ and $S'$ with relative permittivity 
        $\varepsilon_r$.
      \end{itemize}
    \end{columns}

    \framebreak
        
    \hfill TM incident field $\varepsilon_r=1$ \hfill\mbox{}

    \begin{columns}
      \column{0.35\linewidth}
      \includegraphics[width=\linewidth]{results/TM/E_Sp.pdf}

      \includegraphics[width=\linewidth]{results/TM/E_S.pdf}
      
      \column{0.35\linewidth}
      \includegraphics[width=\linewidth]{results/TM/H_Sp.pdf}

      \includegraphics[width=\linewidth]{results/TM/H_S.pdf}
      
    \end{columns}
    
    \framebreak
        
    \hfill TM incident field $\varepsilon_r=4$ \hfill\mbox{}

    \begin{columns}
      \column{0.35\linewidth}
      \includegraphics[width=\linewidth]{results/TMc4/E_Sp.pdf}

      \includegraphics[width=\linewidth]{results/TMc4/E_S.pdf}
      
      \column{0.35\linewidth}
      \includegraphics[width=\linewidth]{results/TMc4/H_Sp.pdf}

      \includegraphics[width=\linewidth]{results/TMc4/H_S.pdf}
      
    \end{columns}
    
    \framebreak

    \hfill TE incident field $\varepsilon_r=1$ \hfill\mbox{}
    
    \begin{columns}
      \column{0.35\linewidth}
      \includegraphics[width=\linewidth]{results/TE/E_Sp.pdf}

      \includegraphics[width=\linewidth]{results/TE/E_S.pdf}
      
      \column{0.35\linewidth}
      \includegraphics[width=\linewidth]{results/TE/H_Sp.pdf}

      \includegraphics[width=\linewidth]{results/TE/H_S.pdf}
      
    \end{columns}

    \framebreak

    \hfill TE incident field $\varepsilon_r=1.5$ \hfill\mbox{}
    
    \begin{columns}
      \column{0.35\linewidth}
      \includegraphics[width=\linewidth]{results/TEc15/E_Sp.pdf}

      \includegraphics[width=\linewidth]{results/TEc15/E_S.pdf}
      
      \column{0.35\linewidth}
      \includegraphics[width=\linewidth]{results/TEc15/H_Sp.pdf}

      \includegraphics[width=\linewidth]{results/TEc15/H_S.pdf}
      
    \end{columns}
    




  \end{frame}

  
  % %%%%%%%%%%%%%%%%%%%%%%%%%%%%%%%%%%%%%%%%%%%%%%%%%%%%%%%%%%%%%%%

    \begin{frame}[fragile,allowframebreaks]{\GreenTEw}

    
    \begin{block}{Ewald Method in HOFEM}
    \begin{columns}[T]
      \column{0.50\textwidth}
      \includegraphics[width=\textwidth,clip=true,trim=70 350 70 80]{Tesis_Daniel_Ewald_1} 
      \column{0.45\textwidth}
      \begin{lstlisting}[style=myFORTRANcodeS,basicstyle=\ttfamily\tiny]
!Distancia de la celda unidad en X
 Dx = SQRT(DOT_PRODUCT(PBC_structure%offset_vector(1,:),PBC_structure%offset_vector(1,:)))
 !Distancia de la celda unidad en Y
 Dy = SQRT(DOT_PRODUCT(PBC_structure%offset_vector(2,:),PBC_structure%offset_vector(2,:)))
    
 E = SQRT(PI/Dx/Dy)

 !Sum using Ewald transformation
 DO m=-2,2
    DO n=-2,2

    !Calcular alpha_mn
    alpha_mn = SQRT(CMPLX((PI*m/Dx)**2 + (PI*n/Dy)**2 + (PI*m/Dx)*Kx + &
    (PI*n/Dy)*Ky + (1.0_DBL/4.0_DBL)*(Kx**2+Ky**2-K**2),0.0_DBL,KIND=DBL));
      \end{lstlisting}
    \end{columns}
  \end{block}


  \begin{block}{Ewald 1D periodicity}
    \begin{itemize}
    \item Naively we thought if was simply setting
      either $m=0$ or $n=0$
    \item \alert{But it is NOT} \ldots
%      \hyperlink{Ewald1D}{click here to go to section on Ewald 1D}
      \hyperlink{Ewald1D}{see next section on Ewald 1D}
    \end{itemize}
  \end{block}

  \end{frame}

  % %%%%%%%%%%%%%%%%%%%%%%%%%%%%%%%%%%%%%%%%%%%%%%%%%%%%%%%%%%%%%%%
