%%%%%%%%%%%%%%%%%%%%%%%%%%%%%%%%%%%%%%%%%%%%%%%%%%%%%%%%%%%%%%%%%%%%%%%%%
% $Id$
% %%%%%%%%%%%%%%%%%%%%%%%%%%%%%%%%%%%%%%%%%%%%%%%%%%%%%%%%%%%%%%%%%%%%%%%%%
%
% Set de slides dedicadas a MPI related issues
%
% %%%%%%%%%%%%%%%%%%%%%%%%%%%%%%%%%%%%%%%%%%%%%%%%%%%%%%%%%%%%%%%%%%%%%%%%%

\begin{frame}[fragile,allowframebreaks]{EMULATION\_2D MPI Broadcast}

  \begin{itemize}
  \item From FREQ-SWEEP to POST-MODE we write and read from disk 
%
    \begin{block}{Write to Disk}
    \begin{lstlisting}[style=myFORTRANcodeS,basicstyle=\ttfamily\footnotesize]
   SUBROUTINE write_EMULATION_2D_data_structure(filename)
     
     CHARACTER(LEN=*), INTENT(IN) :: filename
     INTEGER:: ierror,fileunit
     
     OPEN(NEWUNIT=fileunit, FILE=filename, STATUS="REPLACE", ACTION="WRITE", FORM="UNFORMATTED",IOSTAT=ierror)
     IF (ierror /= 0) THEN
        CHKERRQ(ierror,' write_EMULATION_2D_data_structure: Error opening file')
     ENDIF
     WRITE(fileunit,IOSTAT=ierror)  EMULATION_2D_scattering_structure
     
     CLOSE(fileunit)
     
   END SUBROUTINE write_EMULATION_2D_data_structure
 \end{lstlisting}
\end{block}
 
   \framebreak % %%%%%%%%%%%%%%%%%%%%%%%%%%

   \begin{block}{Read from Disk}
   \begin{lstlisting}[style=myFORTRANcodeS,basicstyle=\ttfamily\footnotesize]
   SUBROUTINE read_EMULATION_2D_data_structure(filename)

     CHARACTER(LEN=*), INTENT(IN) :: filename
     INTEGER:: ierror,fileunit

     OPEN(NEWUNIT=fileunit, FILE=filename, STATUS="OLD", ACTION="READ", FORM="UNFORMATTED",IOSTAT=ierror)     
     IF (ierror /= 0) THEN
        CHKERRQ(ierror,' read_EMULATION_2D_data_structure: Error opening file')
     ENDIF
     READ(fileunit,IOSTAT=ierror)  EMULATION_2D_scattering_structure
     
     CLOSE(fileunit)

   END SUBROUTINE read_EMULATION_2D_data_structure
 \end{lstlisting}
\end{block}

\framebreak % %%%%%%%%%%%%%%%%%%%%%%%%%%

\item MPI communication between processes within POST-MODE
  \begin{itemize}
  \item At present it is simply by all MPI processes reading the
    EMULATION\_2D data structure from disk (and not only process~0)

  \item \alert{Use of custom MPI datatypes} (simple prototype under
    test at present)
    \begin{itemize}
    \item One MPI call: low latency, readability of the code
    \item Maintenability of the code: routines defined on each module
      to create MPI datatype associated to each derived datatype.
    \end{itemize}
  \end{itemize}

  
   \framebreak % %%%%%%%%%%%%%%%%%%%%%%%%%%

   \begin{block}{Fortran Derived DataType}
     \begin{lstlisting}[style=myFORTRANcodeS,basicstyle=\ttfamily\footnotesize]
   ! Derived type to test MPI_Type_create_struct
   TYPE :: my_derived_type_def
      LOGICAL :: mylogical = .TRUE.
      CHARACTER(len=4) :: mystring = 'hola'
      REAL(KIND=DBL) :: myreal = 1.0_DBL
   END TYPE my_derived_type_def

   TYPE(my_derived_type_def) :: my_derived_type   
     \end{lstlisting}
   \end{block}
   
   \framebreak % %%%%%%%%%%%%%%%%%%%%%%%%%%
   
\begin{block}{MPI datatype creation (MPI\_Type\_create\_struct) \insertcontinuationtext}   
\begin{lstlisting}[style=myFORTRANcodeS,basicstyle=\ttfamily\footnotesize]
   !! Creamos el MPI Datatype que se use para pasar el tipo derivado
   !! entre procesos MPI

   CALL MPI_GET_ADDRESS(my_derived_type,address(1), ierror)
   CALL MPI_GET_ADDRESS(my_derived_type%mylogical,address(2), ierror)
   CALL MPI_GET_ADDRESS(my_derived_type%mystring,address(3), ierror)
   CALL MPI_GET_ADDRESS(my_derived_type%myreal,address(4), ierror)

   DO i=1,COUNT
      ! MPI-3.0 & former: disp = iaddr(i+1)-iaddr(i)
      displacements(i) = MPI_Aint_diff(address(i+1),address(i))
   ENDDO

   typelist(1) = MPI_LOGICAL
   typelist(2) = MPI_CHARACTER
   typelist(3) = MPI_DOUBLE_PRECISION
 \end{lstlisting}
\end{block}



\begin{block}{MPI datatype creation (MPI\_Type\_create\_struct) \insertcontinuationtext}   
\begin{lstlisting}[style=myFORTRANcodeS,basicstyle=\ttfamily\footnotesize]
   block_lengths(1)=1
   block_lengths(2)=4
   block_lengths(3)=1

   ! build the derived data type
   call MPI_Type_create_struct(COUNT,block_lengths,displacements,&
        typelist, my_mpi_derived_type_def,ierr)
   if (ierr /= 0 ) then
      print *, 'got an error in type create: ', ierr
      call MPI_Abort(MPI_COMM_WORLD, ierr, ierr)
   endif

     \end{lstlisting}
   \end{block}
   
   \framebreak % %%%%%%%%%%%%%%%%%%%%%%%%%%
   
\begin{block}{Use of MPI datatype in MPI calls}   
\begin{lstlisting}[style=myFORTRANcodeS,basicstyle=\ttfamily\footnotesize]
   ! commit it to the system, so it knows we ll use it
   ! for communication
   call MPI_TYPE_COMMIT(my_mpi_derived_type_def)
   if (ierr /= 0 ) then
      print *, 'got an error in type commit: ', ierr
      call MPI_Abort(MPI_COMM_WORLD, ierr, ierr)
   endif

   ! use it
   call MPI_BCAST(my_derived_type,1,my_mpi_derived_type_def,0,MPI_COMM_WORLD,ierr)
 \end{lstlisting}
\end{block}

 \begin{block}{Free MPI datatype created}   
\begin{lstlisting}[style=myFORTRANcodeS,basicstyle=\ttfamily\footnotesize]
   ! We free the MPI Datatype created
   call MPI_TYPE_FREE(my_mpi_derived_type_def,ierr)
 \end{lstlisting}
\end{block}


\end{itemize}

\end{frame}

% %%%%%%%%%%%%%%%%%%%%%%%%%%%%%%%%%%%%%%%%%%%%%%%%%%%%%%%%%%%%%%%%%%%%%%%%%

\begin{frame}[fragile,allowframebreaks]{MPI Fortran interfaces}

 \begin{block}{USE mpi\_f08}
  \begin{lstlisting}[style=myFORTRANcodeS,basicstyle=\ttfamily\footnotesize]
    USE mpi_f08
   !! It requires compile-time argument checking with unique MPI
   !! handle types and provides techniques to fully solve the
   !! optimization problems with nonblocking calls. This is the only
   !! Fortran support method that is consistent with the Fortran
   !! standard (Fortran 2008 + TS 29113 and later). This method is
   !! highly recommended for all MPI applications.
 \end{lstlisting}
\end{block}


 \begin{block}{USE mpi}
  \begin{lstlisting}[style=myFORTRANcodeS,basicstyle=\ttfamily\footnotesize]
    USE mpi 
   !! It requires compile-time argument checking. Handles are defined
   !! as INTEGER. This Fortran support method is inconsistent with the
   !! Fortran standard, and its use is therefore not recommended.
 \end{lstlisting}
\end{block}


 \begin{block}{INCLUDE 'mpif.h'}
  \begin{lstlisting}[style=myFORTRANcodeS,basicstyle=\ttfamily\footnotesize]
    INCLUDE 'mpif.h'
   !!  The use of the include file mpif.h is strongly discouraged
   !!  starting with MPI-3.0, because this method neither guarantees
   !!  compile-time argument checking nor provides sufficient
   !!  techniques to solve the optimization problems with nonblocking
   !!  calls, and is therefore inconsistent with the Fortran standard.
\end{lstlisting}
\end{block}

\end{frame}

% %%%%%%%%%%%%%%%%%%%%%%%%%%%%%%%%%%%%%%%%%%%%%%%%%%%%%%%%%%%%%%%%%%%%%%%%%
