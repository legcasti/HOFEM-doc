% $Id$

% %%%%%%%%%%%%%%%%%%%%%%%%%%%%%%%%%%%%%%%%%%%%%%%%%%%%%%%%%%%%%%%%%%%%%

% Estilo de los últimos años con linea azul al borde

% \mode<presentation>
% {
% %  \usetheme{Warsaw}
%   \usetheme{Boadilla}

%   \setbeamercovered{transparent}
%   % or whatever (possibly just delete it)

% \setbeamertemplate{navigation symbols}{}

% }

% \useoutertheme{uc3mcourse_updated2020} % curso de uc3m

% %%%%%%%%%%%%%%%%%%%%%%%%%%%%%%%%%%%%%%%%%%%%%%%%%%%%%%%%%%%%%%%%%%%%%

% Estilo de Sergio en traspas Radar, campos, etc

\usetheme{uc3m}


% %%%%%%%%%%%%%%%%%%%%%%%%%%%%%%%%%%%%%%%%%%%%%%%%%%%%%%%%%%%%%%%%%%%%%


%\setbeameroption{show notes on second screen=bottom}

%\setbeamersize{text margin right=0.1\linewidth,text margin left=0.1\linewidth}

\setbeamertemplate{frametitle continuation}[from second]

% \usepackage[spanish,es-noquoting,es-nolists,es-noshorthands]{babel}
%  \decimalpoint % utilizamos punto como separador decimal
%  \deactivatetilden % Desactiva abreviaciones ~n y ~N que dan muchos problemas.
% % %                Por ejemplo, con abreviaciones de bibtex como M.~N. Vouvakis
\usepackage[english]{babel}
% or whatever

% Tikz y derivados
%\input{preamble_tr_tikz}

%\usepackage[latin1]{inputenc}
\usepackage[utf8]{inputenc}
% or whatever

%\usepackage[T1]{fontenc}
% Or whatever. Note that the encoding and the font should match. If T1
% does not look nice, try deleting the line with the fontenc.
%\usepackage{times}
%\usepackage{lmodern}


\usepackage{amsmath}
\usepackage{amssymb}
%\usepackage{amscd} % conmutative diagrams
\usepackage{latexsym}

\usepackage{pifont}


\usepackage{verbatim}
\usepackage{listings}

  \lstdefinestyle{myFORTRANcode}
    {language=[90]Fortran,
    basicstyle=\ttfamily\footnotesize,
    keywordstyle=\color{blue}\ttfamily,
    stringstyle=\color{brown}\ttfamily,
    commentstyle=\color{red}\ttfamily,
    tabsize=2,
%    frame=single,
%    backgroundcolor=\color{yellow},
    breaklines=true,breakatwhitespace=true,prebreak=\space\&
  }
  \lstdefinestyle{myFORTRANcodeS} % S as "small"
    {language=[90]Fortran,
    basicstyle=\ttfamily\scriptsize,
    keywordstyle=\color{blue}\ttfamily,
    stringstyle=\color{brown}\ttfamily,
    commentstyle=\color{red}\ttfamily,
    tabsize=2,
%    frame=single,
%    backgroundcolor=\color{yellow},
    breaklines=true,breakatwhitespace=true,prebreak=\space\&
  }

  \lstdefinestyle{myFORTRANcodeOpenMP}
    {language=[90]Fortran,
    basicstyle=\ttfamily\footnotesize,
    keywordstyle=\color{blue}\ttfamily,
    stringstyle=\color{brown}\ttfamily,
    commentstyle=\color{red}\ttfamily,
    morecomment=[l]{\$},
    tabsize=2,
%    frame=single,
%    backgroundcolor=\color{yellow},
    breaklines=true,breakatwhitespace=false,prebreak=\space\&
  }


\usepackage{graphicx}
\usepackage{subfig}
%\usepackage{subfigure}


\usepackage{xmpmulti}

\usepackage{notacion}

% \usepackage{def_listas}

\usepackage{pdfpages}

\usepackage{multimedia}

\usepackage{array}  % Para tablas
\newcolumntype{C}{>{$}c<{$}} % Identificadores de columna en math mode
\newcolumntype{L}{>{$}l<{$}}
\newcolumntype{R}{>{$}r<{$}}

%\setlength{\arrayrulewidth}{0.5mm}
%\setlength{\tabcolsep}{18pt}
\setlength{\tabcolsep}{10pt}
\renewcommand{\arraystretch}{1.5}


\newcommand{\vbs}{\vspace{\baselineskip}}
\newcommand{\vbss}{\vspace{0.5\baselineskip}}
\newcommand{\vbsf}{\vspace*{\baselineskip}}
\newcommand{\vbssf}{\vspace*{0.5\baselineskip}}
\newcommand{\FIGURA}[1]{\vspace*{2\baselineskip} \centering{FIGURA \\ #1} \vspace*{2\baselineskip} }


\newcommand{\crule}{{\color{cyan} \hrule} }

\setcounter{tocdepth}{4}

% El paquete enumitem redefine las listas de Beamer:
%\usepackage{enumitem}
% \setitemize{label=\usebeamerfont*{itemize item}%
%    \usebeamercolor[fg]{itemize item}
%    \usebeamertemplate{itemize item}}

% Alternativa:
\newcounter{saveenumi}
\newcommand{\seti}{\setcounter{saveenumi}{\value{enumi}}}
\newcommand{\conti}{\setcounter{enumi}{\value{saveenumi}}}
\resetcounteronoverlays{saveenumi}



